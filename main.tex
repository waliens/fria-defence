\documentclass{beamer}

%\usetheme{Boxes}

\usepackage[utf8]{inputenc}
\usepackage[frenchb]{babel}
\usepackage{verbatim}
\usepackage{graphicx}
\usepackage{color}
\usepackage{hyperref}
\usepackage{verbatim}
\usepackage{url}
\usepackage{moreverb}
\usepackage{fancyvrb}
\usepackage{minted}
\usepackage{algpseudocode}
\usepackage{natbib}
\usepackage{eulervm}
\usepackage{auto-pst-pdf}
\usepackage{pst-plot}
\usepackage{multirow}
\usepackage{subfigure}


\hypersetup{colorlinks=true, linkcolor=black, urlcolor=blue}
\usetheme{boxes}
\beamertemplatenavigationsymbolsempty
\setbeamertemplate{sections/subsections in toc}[circle]
\setbeamertemplate{footline}[frame number]
\setbeamertemplate{itemize items}[circle]
\setbeamertemplate{itemize subitem}[square]

\title{
{\bf Tirer avantage de données auxiliaires et d'interactions utilisateurs afin d'apprendre avec peu de données.} \\
{\footnotesize \bf Leveraging user-interaction and auxiliary data to learn from small data.}
}
\author{Romain Mormont}
\institute{Unité systèmes et modélisation, \\ Département d'électricité, électronique et informatique, \\ Université de Liège, Belgique}
\date{\today}  

\newcommand{\todo}[1]{\textcolor{red}{[TODO] #1}}

\definecolor{lightgreen}{rgb}{0.0,0.8,0.0}
\definecolor{lightblue}{rgb}{0.3,0.8,1.0}
\definecolor{lightred}{rgb}{0.874,0.180,0.105}
\definecolor{gray}{rgb}{0.4,0.4,0.4}
\definecolor{lightgray}{rgb}{0.8,0.8,0.8}
\definecolor{shadecolor}{rgb}{0.9,0.9,0.9}
\newrgbcolor{mygreen}{.00 .5 .00}
\newrgbcolor{myyellow}{.6 .6 .00}

\DeclareMathOperator*{\argmax}{arg\,max}


\newrgbcolor{mygreen}{.00 .5 .00}
\newcommand{\X}[1]{\textcolor{blue}{#1}}
\newcommand{\y}[1]{\textcolor{red}{#1}}
\newcommand{\model}[1]{\textcolor{mygreen}{#1}}
\newcommand{\loss}[1]{\textcolor{lightblue}{#1}}

\begin{document}
\setbeamertemplate{caption}{\raggedright\insertcaption\par}
 % \renewcommand{\inserttotalframenumber}{20}

% Title page ==================================================================

\begin{frame}
\titlepage
\end{frame}

\begin{frame}{Contexte}
	\begin{itemize}
		\item Le \textbf{machine learning} (ML) a récemment connu plusieurs succès commerciaux et industriels (DeepMind, voitures auto-pilotées Google et Tesla, IBM Watson,...)
		\item Ces succès sont souvent dûs à deux facteurs: 
		\begin{itemize}
			\item \textbf{Abondance de données} (\textit{big data})
			\item \textbf{Grande puissance de calcul}
		\end{itemize}
		\item Ces \textbf{critères sont essentiels pour appliquer la plupart des méthodes de ML} 
	\end{itemize}
\end{frame}

\begin{frame}{Problème}
	\begin{itemize}
		\item \textbf{Ces critères ne sont pas toujours remplis}.
		\item Les données disponibles ne le sont pas toujours quantité suffisantes. On peut alors parler de \textbf{small data} !
	\end{itemize}
	\vspace{1cm}
	\begin{center}
		\large
		\textbf{Small data}: "la quantité de données n'est pas grande assez relativement à la tâche à résoudre sans pour autant être petite de manière absolue" 
	\end{center}
\end{frame}

\begin{frame}{Applications \textit{small data}}
	Est-ce que je présente des applications ici, où j'attends le case study ? 
\end{frame}

\begin{frame}{Objectifs}
	\begin{center}
		\large
		Explorer et développer de nouvelles méthodes applicables aux problèmes \textit{small data}.
	\end{center}
	
	\vspace{0.5cm}
	Pistes et questions de recherche: 
	\begin{enumerate}
		\item Comment \textbf{intégrer l'humain aux procédures d'apprentissage} afin d'en améliorer les performances ?
		\item Comment \textbf{exploiter des données auxiliaires} disponibles afin d'atteindre le même objectif ? 
	\end{enumerate}
\end{frame}
 
\begin{frame}{Question 1: humain dans la boucle ?}
	
	TODO: Illustration of the "human-in-the-loop" with diagrams ? 
	(one with ML expert, one with domaine expert interacting with algo?)
	
\end{frame}

\begin{frame}{Question 2: données auxiliaires ?}
	
	TODO: Again with an illustration ? Transfert learning, imprecise/precise data, multi-modal images,...
	
\end{frame}

\begin{frame}{Cas d'étude}
	Illustration cytomine \\
		$\Rightarrow$ rare object detection and categorization in high-resolution tissue images \\
		$\Rightarrow$ en production peut être utilisé pour collecter du feedback utilisateur \\
\end{frame}

\begin{frame}{Autres applications (?)}
 	Détection de défaut image de pièces ??  \\
 	Pepite ?? 
\end{frame}

\begin{frame}
\begin{center}
	Merci pour votre attention ! \\ 
	Des questions ? 
\end{center}
\end{frame}


\begin{frame}{Backup slides}
TODO: \\
\begin{itemize}
	\item calendrier, 
	\item résultats du TFE, 
	\item applications à moyen ou long terme, 
	\item publications récentes…
	\item théorie (ML,...)
\end{itemize}
\end{frame}
\end{document}
